\section{How does git work?}

To understand how git works we need to know some basic commands.
As any other VCS git stores its data in the repository directory.
There are two major ways to get a repository.
One would be by simply creating a new empty git repository in your directory of choice:
\begin{lstlisting}
$ git init
\end{lstlisting}
This command creates a new subdirectory .git/ where all your version control
data is stored. \cite{gitinternals2008} p 55 \\
The second is to clone an existing repository over an URL. Possible protocols
are SSH HTTPS HTTP or GIT. Of course on the target location must host a git
repository:
\begin{lstlisting}
$ git clone git://github.com/rails/rails.git
\end{lstlisting}
Git will will clone the whole ruby on rails project from github and we're
ready to get to work. \cite{gitinternals2008} p 56

\subsection {Basic commands}

After we have a new repository there are several basic commands one should know
working with git:
\begin{lstlisting}
$ touch README
$ git status
$ git add README
$ git commit -m "Added README file"
\end{lstlisting}

This code produces an empty README file. The status command displays any change
of the repository. There we can see that the README file has been added and we
track this file simply by telling git to add the file. Additionally git now
stages the file, but we are going to cover the staging feature later in the next
chapter. Then we commit the changes with a short and descriptive message. Git
now stores the new file in the repository and index. The master branch, 
which is the default branch, is pointing towards this new state of the repository. \cite{gitinternals2008} p 55 \\




To collaborate with other people it is very important to share your work or get
other work over the internet or just over a network. You can do that easily with
a provider like github. Cloneing a repository is that easy:



This command  \cite{gitpro2009}
%page missing - chapter Getting a Git Repository
Additionally git creates a new remote host called origin.

After there has been commited a huge amount of work in the local ruby on rails
reposiotry we can share the work telling git:

\begin{lstlisting}
$ git push origin master 
\end{lstlisting}

This command will only succeed if you have the appropriet access rights to the
project. But if we had them, the master parameter
tells git to take the default branch and transmit it to the host. We could also
specify any other branch we already have. The origin parameter is translated and
points to the github server.









\subsection {What makes git to a DVCS?}

Nearly all commands that git offer you are local operations. 
The whole history of your project is contained in your local repository. 
Also all contributors of one project have that history with local changes. 
Therefor no repository has more information about the project than the other.
That is what git does and this makes git to a distributed verion control system.
