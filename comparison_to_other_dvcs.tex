\section {Comparison to other DVSSs}

\subsection{Snapshots instead of differeces}

Git uses a unique store model compared to other common DVCS like Baazar or Mercurial. 
In tradional distributed or normal version control the system keeps differences between revision A and B. 
These changes are saved and marked with a revision number. 
So you can, at any time, switch back to any revision and rebuild the complete content.
This does have some major downsides which we will see later. \\
Git does handle it quite different. It takes socalled snapshots of every new version. 
So if file A changes in your repository the next commit git will take a new snapshot of this file and all other files. 
To reduce overhead not changed files are linked to the identical file in the previous verion. 
It creates a new minifile system rather than a VCS, enabling you all the features normal version 
control would give you and gives you some additional on top of it. \cite{gitpro2009} 