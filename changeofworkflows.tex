\section {Change of Workflows}

Before Distributed Version Control Systems (DVCS) were introduced, Centralized Version Control Systems (CVCS) were the common system to handle team development.
These traditional CVCS were built up in the following way: One central server stores the content of the project, along with a history of the changes made. This means that the complete content (all changes) have to pass the central server. In other words: One developer commits changes (to the server), and all other developers have to update to this version from the server.

//INCLUDE PICTURE - CENTRALIZED VCS

Although this system might look simple and effective at the first place, it has quite a few drawbacks. Let's just assume one developer commits corrupted content (not working source-code, corrupt files, ...).
The other developers will update this version and recognize that what they updated is corrupted, but unfortunately it is already too late at that point of time.
Another problem CVCS has is the lack of support for experimental development. To do things like these, you are almost forced to store a local version where you are experimenting,
but as soon as the developer needs help, it is simply impossible with a centralized setup.


DVCS work in a different way. The main difference compared to the centralized systems is, that there is no central server storing the content.
It is a peer-to-peer setup. Every developer stores the complete project, along with the change history, locally. 

//INCLUDE PICTURE - DISTRIBUTED VCS

Every developer works locally on the
project and is able to commit his changes. Other developers - and here is one of the main advantages compared to a centralized system - can pull these changes,
but they are not required to do so!
The problem with experimental development is also not present in DVCS. Every developer can create as many so called "`Branches"' of the project as he wants. This way, he can simply commit the branch. If other developers like to, they can merge the content from the different branches. Once again: They are not forced to do so.

\cite{branchingstrategies2010}