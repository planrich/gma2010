\section{Explanation of terms}

In this chapter we want to explain some of the phrases, words and terms used through out the article.

\subsection{Repository}

A repository in DCVS terms is a folder which can contain sub folders and files of all categories. 
This folder is tracked by the DCVS so there is a history for every file whether it was renamed, deleted or changed. 
How these changes are tracked is different from system to system. CVS for example “isolates” 
files and handles them individually. So each file has its own version number. If you change a file, CVS will 
save the delta (i.e. what has changed) and give the file a new number. SVN does it a little bit 
different. If you change one file, the whole repository gets a new version number, usually count from 1, 2 to n. 
In SVN terms such a version of the repository is referred to as a “revision”.

Git also handles the repository as a whole. It saves each new file version as a whole and not just the delta as SVN and CVS does. 
In Git there are no incrementing version numbers, instead git uses the SHA-1 checksum. For each file there is such a 
checksum and the checksum which represents a certain revision is a sum of these checksums. 
This explanation might not be 100 percent correct but its a good approximation for those unfamiliar with version control.


\subsection{Commit}

Committing means that you make all your changes you have made to some files inside the repository permanent. 
After committing you have a new revision of your repository. Its possible to move back and forth between commits so you can work 
from an older version if you have screwed something up.

Normally with each commit there is saved some meta information like, who has committed and what has been done since the last commit. 
It's good practice to give detailed information about what files have changed and the purpose of these changes, so others working 
on the repository know whats going on.


\subsection{Branch}

Branching means that you diverge from your main line and continue working without messing up the main line of development. 
Branches are a feature where git really shines because they are very lightweight and really fast compared to other VCS.


\subsection{Merge}

When you want to have changes made in a branch also applied in another branch, you have to merge them. 
Because its very easy to branch and merge with git its very common to have a branch for each feature or 
bug and merge them back into a branch which holds the stable code when they are done.

\subsection{Tagging}

Tagging means that you mark a certain version or revision with a tag. 
Assigning a tag to a version is just for convenience, normally its used to tag important project milestones. 
For example if your application which you are developing reaches version 1.5, you will assign a tag named “v1.5” 
to the commit or revision which was used to build the “application 1.5” or which represents the “application 1.5”. 
So it will be very easy to find the commit you need in the future if you want an older version of your application.