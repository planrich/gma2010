\section{Why revision control} \label{whyrevisioncontrol}

\zitat{Revision control is the process of managing multiple versions of a piece of information.}
\cite[chapter 1]{hgbook2009}

Many people are not aware of the fact that they do revision control in their daily work. For example they create a document or write a program and send it via E-Mail to another person. This person may change the document and send it back. Another example is when someone is working on a document: Every time he saves it he may give the document a new name with an incrementing number so that he has a history of his work.

All this kind of work can be referred to as revision control. It helps people to keep track of what they have done.

In the history of computer science many systems were designed and implemented which automated this process of revision control. Some of them turned out to work very well, others did not.

Advantages of using a Revision Control System (RCS):
\begin{itemize}
\item People have the possibility to work on the same project at the same time. They can even work on the same file simultaneously. The RCS will handle conflicts if any occur.
\item If bugs or errors were introduced, it is possible to revert the project to a previous version or stage.
\item The system will keep track of each step the project takes. It will log who has made changes, how these changes look like and when they were made.
\end{itemize}

As seen above, using such a system may help organizing projects of different size.
