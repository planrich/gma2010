\section {A brief history of Distributed Version Control Systems}

Before Distributed Version Control Systems (DVCS) were introduced, Centralized Version Control Systems (CVCS) were the common system to handle team development.
These traditional CVCS were built up in the following way: One central server stores the content of the project, along with a history of the changes made. This means that the complete content (all changes) have to pass the central server. In other words: One developer commits changes (to the server), and all other developers have to update to this version from the server.

//INCLUDE PICTURE - CENTRALIZED VCS

Although this system might look simple and effective at the first place, it has quite a few drawbacks.


\cite{branchingstrategies2010}