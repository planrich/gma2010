\section {A brief history of Distributed Version Control Systems} \label{dvcshistory}

The history of Distributed Version Control Systems began with BitKeeper, 1997-1999, by Larry McVoy. BitKeeper was (and still is!) already fully decentralized, able to merge as well as able to store a history. In 2002, BitKeeper became very successful when Linus Torvalds used BitKeeper to handle the Linux Kernel development.

History continues with GNU Arch in year 2001/2002. It was made by Tom Lord, that's why it is also referred to as \emph{TLA} - Tom Lord's Arch. Development stopped in 2005, but Arch nowadays has many siblings, for instance ArX and Bazaar.

Around 2003, a lot of new DVCS projects came up, including Monotone, Darcs and Codeville. All of them were based on the ideas of GNU Arch and also influenced each others in their development.

Later, in 2005, Git and Mercurial came up. Mercurial was widely inspired by Monotone. Git by Linus Torvalds is based on the ideas of BitKeeper, but Git is definitely not a BitKeeper clone. Just the basic ideas are the same. Git is free and open source.

The latest (November 2010) most popular DVCS is Fossil, initally launched in 2006. Fossil supports a few features that weren't used before like distributed bug tracking, a distributed wiki and a distributed blog, everything within a package integrated in Fossil. Fossil is cross-platform, also free software, but not open source.
\cite{understandingvcs} \cite{fossilhomepage}
