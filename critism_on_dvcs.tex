\section{Critism on Distributed Version Control Systems}\label{critismondvcs}

Despite all the advantages Distributed Version Control Systems provide, there's also heavy critism going on. Hyrum Wright, president of the Subversion Corporation clearly states a downside of DVCS: \textit{"`In my experience, when a team moves to a Distributed Version Control Tool, they are not really doing that for workflow purposes. Their workflows remain the same. They are doing it because they appreciate the tooling, the features that are available in those tools for use in their team. They still usually have a centralized location where the code lives, very much like a central Subversion repository."'} \cite{subversionandgit}\newline

Hyrum Wright confronts the DVCS community with an argument that is justified. A large number of DVCS projects additionally use a Subversion repository just to store a main version of the product, a version representing the reference point. This is a definite indicator to what Hyrum Wright stated - people are just using DVCS for the features available in DVCS tools, not because they appreciate the workflow. Hyrum Wright goes even further as he states that Subversion learned from DVCS and will soon provide features in SVN so far just available in DVCS. \cite{subversionandgit} The goal that SVN is trying to reach is obvious: If Subversion has no disadvantages tool-wise and the workflows are anyways less decisive for people, the SVN community will grow dramatically.

Another critism is directly confronting the DVCS workflow. In opposite to the DVCS community, the CVCS community states that the DVCS workflows are not only confusing, but also a big problem. Their statement says that any growing DVCS project will sooner or later have problems creating a final version. Every user has different branches and outdated versions of the project. At some point of time, the team loses control over the content and they are not able to merge a final version, because there is no real reference point anymore. 